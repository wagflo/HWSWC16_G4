\documentclass{beamer}
\usetheme{Berlin}  %% Themenwahl
\usecolortheme{beaver}

\usepackage{listings}
\usepackage{graphicx}
\usepackage{hyperref}
\usepackage[utf8]{inputenc}
\usepackage{amssymb}
\usepackage{amsmath}
\usepackage{esvect}
%\usepackage{mcode}
 
\title{Ray Tracing Optimization}
\author{Kashofer, Radschek, Wagner}
\date{\today}

%\section{Foliensection}
%\begin{frame} %%Eine Folie
%	\frametitle{Folientitel} %%Folientitel
%	Das ist eine Dummy-Section
%\end{frame}

\begin{document}
\maketitle
\frame{\tableofcontents[currentsection]}
 
\section{Profiling}
\begin{frame} %%Eine Folie
	\frametitle{Auführungszeiten} %%Folientitel
  	\begin{itemize}
		\item am Anfang: $45 \dfrac{sec}{Frame}$
		\item C-Code Anpassungen: $43 \dfrac{sec}{Frame}$
		\item fix\_mul16 durch ci\_mul in Hauptfunktion getauscht: $36 \dfrac{sec}{Frame}$
		\item fix\_mul16 ruft generell ci\_mul auf: $17 \dfrac{sec}{Frame}$
	\end{itemize}
	$\quad$\\
	$\rightarrow$ Verlagern von SW auf HW bringt benötigte Geschwindigkeit
\end{frame}

\section{Optimierung}
\begin{frame} %%Eine Folie
	\frametitle{Überblick} %%Folientitel
	Lorem ipsum
\end{frame}

\section{Ausblick}
\begin{frame} %%Eine Folie
	\frametitle{Überblick} %%Folientitel
	Set dolor
\end{frame}

\end{document}