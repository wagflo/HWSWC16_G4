\documentclass{beamer}
\usetheme{Berlin}  %% Themenwahl
\usecolortheme{beaver}

\usepackage{listings}
\usepackage{graphicx}
\usepackage{hyperref}
\usepackage[utf8]{inputenc}
\usepackage{amssymb}
\usepackage{amsmath}
\usepackage{esvect}
%\usepackage{mcode}
 
\title{Ray Tracing Optimization}
\author{Kashofer, Radschek, Wagner}
\date{\today}

%\section{Foliensection}
%\begin{frame} %%Eine Folie
%	\frametitle{Folientitel} %%Folientitel
%	Das ist eine Dummy-Section
%\end{frame}

\begin{document}
\maketitle
\frame{\tableofcontents[currentsection]}
 
\section{Profiling}
\begin{frame} %%Eine Folie
	\frametitle{Rendering time} %%Folientitel
  	\begin{itemize}
		\item original code: $45 \dfrac{sec}{Frame}$
		\item simple C-Code adaptions: $43 \dfrac{sec}{Frame}$
		\item replaced fix\_mul16 by ci\_mul looped functions: $36 \dfrac{sec}{Frame}$
		\item fix\_mul16 calls ci\_mul: $17 \dfrac{sec}{Frame}$
	\end{itemize}

\end{frame}

\begin{frame} %%Eine Folie
	\frametitle{SW code structure} %%Folientitel
	getClosestSphere is a main function and contains a lot of loops\\
	$\quad$\\
	$\rightarrow$ Transition looped functions from SW to HW to achive target speed
\end{frame}

\begin{frame} %%Eine Folie
	\frametitle{HW ressouces} %%Folientitel
	\begin{itemize}
		\item xyz Logic cells
		\item abc M9Ks
		\item uvw DSPs
	\end{itemize}
	$\quad$\\
	$\rightarrow$ Plenty Logic-cells (registers and LUTs), keep number of DSPs and M9Ks low
\end{frame}

\section{Optimization}
\begin{frame} %%Eine Folie
	\frametitle{Overview} %%Folientitel
		\begin{figure}[h]
		\centering
		\fbox{\includegraphics[width = 0.6\textwidth]{pics/frontend.png}}
		\caption{Pipepline 1}
	\end{figure}
\end{frame}
\begin{frame} %%Eine Folie
	\frametitle{Overview} %%Folientitel
	\begin{figure}[h]
		\centering
		\fbox{\includegraphics[width = 0.6\textwidth]{pics/frontend.png}}
		\caption{Pipepline 1}
	\end{figure}
\end{frame}
\begin{frame} %%Eine Folie
	\frametitle{Overview} %%Folientitel
	\begin{figure}[h]
		\centering
		\fbox{\includegraphics[width = 0.6\textwidth]{pics/middle.png}}
		\caption{Pipepline 2}
	\end{figure}
\end{frame}
\begin{frame} %%Eine Folie
	\frametitle{Overview} %%Folientitel
	\begin{figure}[h]
		\centering
		\fbox{\includegraphics[width = 0.6\textwidth]{pics/backend.png}}
		\caption{Pipepline 3}
	\end{figure}
\end{frame}

\section{Future tasks}
\begin{frame} %%Eine Folie
	\frametitle{Future tasks} %%Folientitel
	Set dolor
\end{frame}

\end{document}